\section{Conclusiones y trabajo futuro}

% Conclusiones

\begin{frame}{Conclusiones}

\begin{itemize}[<+->]
\item \textbf{Arquitectura del sistema.} Arquitectura modular, componentes independientes, fácil mantenimiento y desarrollo.
\item \textbf{Simulador térmico.} Simulación de lectura y aumento de temperatura. Control de la temperatura interna del sistema.
\item \textbf{Módulo de telemetría.} Definición e implementación de paquetes periódicos y puntuales. Optimización de espacio, evitando pérdida de precisión. Adaptado para sistemas empotrados.

\end{itemize}

\end{frame}


% Trabajo futuro

\begin{frame}{Trabajo futuro}

Primer año de la Cátedra UC3M - SENER, trabajo a realizar en siguientes iteraciones:

\begin{itemize}
\item \textbf{Integración con Simulink.} Simulación con modelos reales (\emph{model-in-the-loop}).
\item \textbf{Gestión de memoria.} Nuevos paquetes, corrección de errores, nueva funcionalidad (\emph{payloads}).
\item \textbf{Documentación.} Generar documentación del código con Doxygen.
\item \textbf{Sensores/actuadores.} Nuevos componentes que complementen la misión.
\end{itemize}

\end{frame}